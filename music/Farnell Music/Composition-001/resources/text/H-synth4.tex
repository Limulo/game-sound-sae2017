\paragraph{Poly synth}
To make chords we need more than one note. Although we could treat
each synth as a monophonic instrument and build up harmonies by
careful sequencing, sometimes we like to be able to specify block
chords that will be played in a polyphonic way. To do this we first
design a single voice of the instrument. This synth is a variation
on our first sinewave instrument. It has a vibrato
and MIDI note input. Control of the final signal amplitude is
just one multiply, so the instrument is not capable of playing 
separate notes as such, only chords that start and end together.


\paragraph{3 voice synth}
You can see three of the voices arranged with the frequency of each controlled
by the corresponding outlet of the \verb+[unpack]+ unit. \verb+[unpack]+ splits up the list of notes sent
as a chord by the sequencer. We separated the envelope trigger
from the note list using [t l b] and use the bang to
activate the envelope message box. 


\paragraph{ASR envelope}
Notice also that the message to \verb+[vline~]+ has a softer attack
and a significant delay before the decay stage, which
causes the chord to rise in 35ms and hold for a further 165ms, the remainder 
of the delay time, then fall to zero in 100ms each time it
plays. Because 3 voices will be playing at normalised level we need to
divide the output signal by 3 to get it back into a normalised range.