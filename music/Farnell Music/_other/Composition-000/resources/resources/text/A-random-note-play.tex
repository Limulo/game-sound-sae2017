\paragraph{Timebase}
To start with we want to create a regular stream of pulses to trigger events. These are bang messages, which really just means "compute something". To get a source of bang messages we need a metronome object, \verb+[metro]+. When a value of 1 appears on the left inlet of a metronome it switches on and begins to emit bangs. It will turn off again uopn recieving a 0 value. The right inlet of \verb+[metro]+ sets the speed, or rather the period which is in milliseconds. A toggle switch outputs a 1 or 0, so we can start and stop the \verb+[metro 300]+ unit. When started it outputs bang messages at a rate somewhat above 3 per second. This is the most simple timebase, just a regular stream of bang messages. 

\paragraph{Random number}
To turn the bang messages into frequencies we need to make numbers, float values that can be understood by an oscillator or other units. Each bang message recieved by \verb+[random 24]+ causes it to choose a new random integer number of 24 possible random values, in the range 0 to 23. We add this to a fixed value in \verb@[+ 48]@ so that the minimum value is 48, and the maximum value will be $48 + 23 = 71$. If you attach a number box you will see the number changing to show a different float message is on the wire. On a musical keyboard there are about 88 notes, two octaves of which covers 24 keys, so this seems like a good pitch range for melodies, starting on \verb+C3+ and ending on \verb+B4+.  


\paragraph{Oscillator}
Connecting this to the frequency inlet of \verb+[osc~]+ produces an audio signal. The left inlet
of \verb+[osc~]+ sets its frequency in Hz or cycles per second. Both sides of
the \verb+[dac~]+ are fed from the same \verb+[osc~]+ outlet so we get to hear both speakers on 
a stereo sound system. The patch plays loudly and low,
so watch your bass bins. It doesn't produce the correct range of frequency.  A range of 48Hz to 71Hz is right 
down in the bottom, so we must scale it for MIDI note values to produce the right frequencies. 

   