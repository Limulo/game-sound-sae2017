\paragraph{center left delay}
Here is the delay at the heart of most our effects so far. In the feedback
circuit is a filter to limit recirculating signals to the midrange.
\verb+[clip]+ is added defensively to make sure that a feedback greater than
0.9 is never sent by accident.
Delays work using pairs of objects in Puredata, one is to send to
the buffer and the other to read from it. The second parameter of the 
\verb+[delwrite]+ object allocates memory, in milliseconds (times the
current sample rate). The names of reads and writes must match and should
be uniqe to that pair of objects in most uses, although many reads from
the same write is okay for multi-tap reverbs. Time may be varied on
\verb+[delread]+ via the first inlet with a float message, but this
causes unpredictable clicks as you might expect.
Signals arriving at the inlet go first to left and right outlets, and
then the delayed signal appears on the left channel only. You can create
many variations on the basic stereo delay effect each with their own
unique uses and applications. A ping pong is a delay that exchanges left
and right in the feedback, a lcr distributes delays left, center, right, and
so on. 
