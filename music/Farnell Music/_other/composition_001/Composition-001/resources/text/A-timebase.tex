\paragraph{Making a counter}
Let us begin with way to divide time. Place a metronome object set to 125ms (8 beat at 120bpm), add a float
object, \verb+[f]+ and an addition object, \verb@[+]@, and link them as shown so
that the next value held in the float will be its last value plus
one. You can verify that the counter works by attaching a number box
to the float output and a toggle to the left inlet as previously.
Make sure audio processing is turned on. Click the toggle to start the metronome running. Each time the float
gets hit with another bang it advances one step. A trigger unit is
used after the metro to split the bang messages into order, right
to left - so the counter is incremented before the bang is sent. 

\paragraph{Abstraction}
Delete the toggle and number box you are using to test the counter. Add two inlet boxes, one for
the start/stop value and one for the period, and add two outlet boxes at the bottom, one for
regular bang messages and one for float messages from the counter.
We make the timebase an abstraction simply by saving it as a new file in the same
directory as the main patch. We gave this the name \verb+A-timebase+ so when
we use it in another patch we do so by creating a new object \verb+[A-timebase]+.
Abstractions are not the same as sub-patches which we created before by prefixing
a new object with pd like \verb+[pd mysubpatch]+, you must remember to save each abstraction
you edit because it isn't a part of the main patch, the main patch just refers to it.
Unlike sub-patches an abstraction also has its own memory space when loaded, so you can
duplicate abstractions without worry. 
 
