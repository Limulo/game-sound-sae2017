\paragraph{MIDI note convertion}
Here we have an \verb+[osc~]+ object with its frequency inlet connected to a number 
that comes via a \verb+[mtof]+ unit. This converts "midi" to "frequency", where
a midi note is an integer in the range 0 to 127, and frequency is a value
in Hertz between zero and about twelve thousand. 

\paragraph{Number box type 2}
The number boxes in pd-gui can be used either as display devices or as
input devices. Here is shown the type 2 box which can be coloured and
has other useful properties. Moving the number by clicking and pushing your mouse
up and down sends new values to \verb+[mtof]+. If you want finer grain
control than just integers try holding down the shift key before you click.

\paragraph{Message melody}
Some message boxes can be immediately used to compose a melody. This isn't
the perfect way to write music, but you can simply add message boxes and 
click them.


\paragraph{Range}
Listen to the output frequency change as you move the number over the whole
MIDI scale from 0 to 127. Connect another number box to the output of \verb+[mtof]+
and notice the relationship between key number and pitch is not a simple
linear one. Each 12 notes doubles the frequency.

