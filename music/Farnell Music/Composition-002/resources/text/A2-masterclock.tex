\paragraph{metronome}
The metro is always started by default here, you can remove the loadbang 
if you don't want the piece to start automatically. The metro period
is set from the middle inlet and if anything is received on the right
inlet the counter is set to zero.

\paragraph{late and early beats}
You can see how time is made a beat earlier by taking its value
from after the \verb@[+ 1]@ instead of from the float box. You could 
chain these or add arbitrary increments or decrements to the timeline.
If you want a part to play 4 beats ahead add \verb@[+ 4]@ before its time inlet.
Beare of negative times that won't exist when the timebase is zero.

 
\paragraph{dividing time} 
 Dividing time is as simple as using \verb+[/]+. But this raises
 interesting points to do with number lines and quantisation.
 A timeline that is multiplied by an amount is stretched or shrunk
 like a rubber band. If you are using select statements that work
 on integer beat marks then you need to re-quantise with \verb+[int]+
 and \verb+[change]+. If a timeline is doubled then it loses half
 its resolution, but if it is squashed to half then its resolution
 would double, if it were say audio tape, however it halves.
 If we had an integer timeline ${1, 2, 3 ...}$ and divide by 2
 we get ${0.50, 1, 1.50, 2 ...}$, some of the values have become 
 non-integer. Although we can make any floating point number that
 is a function of a periodically updated timebase it
 is sampled at the rate of the timbase itself, which is when 
 message events occur. These two things, the value and the time it
 is sent, are separate things. Using a rounding process two notes must now
 fall into each event slot (pigeonhole principle). Going the other
 way let's take a timeline ${1, 2, 3, 4}$ and multiply it by
 3. We get ${3, 6, 9, 12}$ You can build up
 sophisticated rythmns using \verb+[mod]+, \verb+[int]+,
 \verb+[change]+ and \verb+[div]+, but it's recommended that if
 you study this be aware it depends on the implementation
 of \verb+[int]+, there are other ways of rounding numbers
 that will break your compositions if you translate them or 
 the definition of \verb+[int]+ changes.
