\paragraph{Sequencing the melody}
Wiring up \verb+[select]+ objects is fine for making fixed patterns
like rhythms to store away, but not much use when you want to
change a sequence rapidly without breaking connections. 
This time the sequence is still fixed in a particular order, just 4 steps
that cycle round, but the values at each position are variable. This
is sometimes called an arpeggiator, an 8 or 16 step sequencer found on
many old analogue modular synths.

\paragraph{Unpacking lists}
Instead of sending each of the four notes separately into this 
sequencer abstraction we have just one inlet \verb+[inlet note sequence]+, that carries a list like \verb+[12 34 56 78(+.
The \verb+[select]+ is fed by \verb+[mod 4]+ and sends a bang to each
of 4 outputs in turn. These cause successive \verb+[f]+ boxes to output
the float stored in them. To get the note numbers from the list we 
use \verb+[unpack]+ to split up the list and distribute the numbers,
to the left inlet of each float box. The sequence message is always unpacked
in the same left to right ordering, but remember the values appear at the
outlets of \verb+[unpack]+ in right to left evaluation order just as with
any Puredata object.

\paragraph{Transposition}
With a wire carrying note numbers as individual floats its easy to transpose, here we add
an octave with \verb@[+ 12]@ . The third inlet, \verb+[inlet transpose]+ provides a way to control the
transposition from the outside.