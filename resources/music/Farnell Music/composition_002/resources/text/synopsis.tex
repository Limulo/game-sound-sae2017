\section{Synopsis}

\paragraph{Introduction}
This is the third tutorial in a series on composing electronic music with Puredata. 
Composition000 and Composition001 introduced simple DSP and message sequencing. Here
we consolidate that knowledge with a study no more difficult than the last
part, showing the development of a 'real' piece of music. This might serve as
a good enough template, or inspiration for documenting you own Pd projects, breaking
down each abstraction and explaining your code comments with more detailed text
alongside the diagrams and sound captures. Skip this part if you are focusing on
MIDI composition and go to the next appropriate tutorial. 



\paragraph{Aims}
To complete a piece of music with at least six parts of diverse timbre. Use GUI objects to
make a mixer and reuse abstractions. Explore random choices, selection and sequence in
regard to making a music score. Substitute parameters into lists to make adaptations
to a synthesiser. Create some delay and filter effects.

\paragraph{Methods}
We will add parts in compositional order, as I did when making the original piece.
Starting with some chords and a pad sound we establish the harmonic structure and pace.
A bass part and drums fix the beat and root, then we add some melody and arpeggio
instruments. Each part is built in its own abstraction and some parts are included
many times. The sequencer, synth, and any effects will be built each time before
listening to the result.

\paragraph{New units used}
\begin{itemize}
\item \verb+[vd~]+: Variable delay.
\item \verb+[spigot]+ A tap or valve for messages.
\item \verb@[del]@ Delay a bang message.
\item \verb+[sig~]+ Turn a message into a signal.
\item \verb+[vcf~]+ Better variable bandpass filter.
\item \verb+[delread~]+ Read from a named delay buffer.
\item \verb+[delwrite~]+ Write to a named delay buffer.
\item \verb+[clip~]+ Audio clip, set max and min signal range.
\end{itemize} 